\paragraph*{Übung~7.}
\begin{enumerate}
    \item sich beeinflussen lassen von + Dativ
        \begin{description}
            \item Du solltest dich von Sven nicht beeinflussen lassen, weil er sehr unverantwortlich ist.
        \end{description}
    \item sich bemühen um + Akkusativ
        \begin{description}
            \item Danke schön, aber ich kann mich um meine Noten selbst bemühen.
        \end{description}
    \item sich beschäftigen mit + Dativ
        \begin{description}
            \item Womit beschäftigt sich deine Mutter in ihrer Arbeit?
        \end{description}
    \item sich beteiligen an + Dativ
        \begin{description}
            \item Markus hat sich an der Vorbereitung des Sporttags beteiligt.
        \end{description}
    \item sich engagieren für + Akkusativ / sich einsetzen für + Akkusativ
        \begin{description}
            \item Peter will sich für diese Schenkung engagieren, aber er ist zu jung und braucht eine Erlaubnis seiner Eltern.
        \end{description}
    \item sich entscheiden für + Akkusativ / die Entscheidung trefeen
        \begin{description}
            \item Mein Nachbar hat sich für den Umzug entscheiden.
        \end{description}
    \item sich entschuldigen für + Akkusativ
        \begin{description}
            \item Heute ist mein Bus zu spät gekommen und ich musste mich für die Verspätung beim Lehrer entschuldigen!
        \end{description}
    \item sich erkundigen nach + Dativ
        \begin{description}
            \item Wenn Sie sich nach etwas erkundigen wollen, steht die Touristeninformation zur Ihre Verfügung.
        \end{description}
    \item sich fürchten vor + Dativ
        \begin{description}
            \item Warum fürchtest du dich vor meinem Hund? Er bellt laut, aber er ist sehr freundlich!
        \end{description}
    \item sich informieren über + Akkusativ
        \begin{description}
            \item Entschuldigen Sie, bitte! Wo kann ich mich über die Sehenswürdigkeiten dieser Stadt informieren?
        \end{description}
    \item sich kümmern um + Akkusativ
        \begin{description}
            \item Max ist sehr verantwortlich und er kümmert sich die ganze Zeit um seinen Hund Fifi.
        \end{description}
    \item sich richten nach + Dativ
        \begin{description}
            \item Das war keine gute Entscheidung! Wonach hast du dich gerichtet?
        \end{description}
    \item sich trennen von + Dativ
        \begin{description}
            \item Niemand konnte uns voneinander trennen, weil wir beste Freunde waren.
        \end{description}
    \item sich unterhalten mit + Dativ
        \begin{description}
            \item Während der Party habe ich mich vor allem mit Michael unterhalten.
        \end{description}
    \item sich unterhalten über + Akkusativ
        \begin{description}
            \item Obwohl wir viel Zeit hatten, haben wir uns nur über Programmieren unterhalten.
        \end{description}
    \item sich verlassen auf + Akkusativ
        \begin{description}
            \item Mattias ist mein bester Freund und ich kann mich auf ihn immer verlassen.
        \end{description}
    \item sich verlieben in + Akkusativ
        \begin{description}
            \item Wenn ich sie gesehen habe, habe ich mich in sie im Augenblick verlieben.
        \end{description}
    \item sich vorbereiten für + Akkusativ
        \begin{description}
            \item Heute haben wir eine Prüfung in Mathe, aber ich habe mich dafür nicht vorbereitet.
        \end{description}
\end{enumerate}
\paragraph*{Übung~8.}
\begin{enumerate}
    \item achten auf --- auf die Akustik der Konzertsäle achten
    \item berichten über --- über schöne Orte und fantastiche Konzerte berichten
    \item sich freuen auf --- sich auf beide Aspekte freuen
    \item sich konzentrieren auf --- sich auf musikalische Übungen konzentrieren
    \item reagieren auf --- auf die Musik unterschiedlich reagieren
    \item reden mit \ldots über --- mit älteren Freunden über ihre Reisen reden
    \item träumen von --- von Konzerten in bekannten Konzertsälen träumen
    \item sich unterscheiden von --- sich nich nur durch akustischen Bedingungen voneinander unterschieden
    \item verzichten auf --- auf das Ausgehen mit Freunden verzichten
\end{enumerate}
